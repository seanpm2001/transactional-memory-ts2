%!TEX root = ts.tex
\setcounter{section}{9}
\rSec0[lex]{Lexical conventions}


\rSec1[lex.name]{Identifiers}

In \CppXref{lex.name},
add \added{\tcode{atomic}} to the table of identifiers with special meaning (Table 4).

\rSec0[basic]{Basics}

\setcounter{section}{8}

\rSec1[basic.exec]{Program execution}

\setcounter{section}{9}

\rSec2[intro.execution]{Sequential execution}

Change in \CppXref{intro.execution} paragraph 5 as indicated:

\begin{quote}
\setcounter{Paras}{4}
\setcounter{Bullets1}{3}
\pnum
A \defn{full-expression} is

\begin{itemize}
\item
  ...
\item
  an invocation of a destructor generated at the end of the lifetime of
  an object other than a temporary object (6.7.7) whose lifetime has not
  been extended, \removed{or}
\item
  \added{the start and the end of an atomic block (\ref{stmt.tx}),
  or}
\item
  an expression that is not a subexpression of another expression and
  that is not otherwise part of a full-expression.
\end{itemize}
\end{quote}

\rSec2[intro.multithread]{Multi-threaded executions and data races}
\setcounter{subsubsection}{1}
\rSec3[intro.races]{Data races} 

Change in \CppXref{intro.races} paragraph 6 as indicated:

\begin{quote}
\setcounter{Paras}{5}
\pnum
\removed{Certain} \added{Atomic blocks as well as certain} library
calls \added{may} \emph{synchronize with} other \added{atomic blocks
and} library calls performed by another thread.
\end{quote}

Add a new paragraph after \CppXref{intro.races} paragraph 20:

\begin{quote}
\pnum
\setcounter{Paras}{20}
\setcounter{note}{20}
An atomic block that is not dynamically nested within another atomic
block is termed a \defn{transaction}.
\begin{note}
Due to syntactic constraints,
blocks cannot overlap unless one is nested within the other.
\end{note}
There is a global total order of execution for all transactions.
If, in that total order, a transaction T1 is ordered before a transaction T2,
then

\begin{itemize}
\item
  no evaluation in T2 happens before any evaluation in T1 and
\item
  if T1 and T2 perform conflicting expression evaluations, then the end
  of T1 synchronizes with the start of T2.
\end{itemize}

\begin{note}
If the evaluations in T1 and T2 do not conflict, they might be
executed concurrently.
\end{note}

\pnum
Two actions are \defn{potentially concurrent} if ...
\end{quote}

Change in \CppXref{intro.races} paragraph 21:

\begin{quote}
\setcounter{Paras}{20}
\setcounter{note}{20}
\pnum
...
\begin{note}
It can be shown that programs that correctly use
mutexes\added{, atomic blocks,} and \tcode{memory_order::seq_cst}
operations to prevent all data races and use no other synchronization
operations behave as if the operations executed by their constituent
threads were simply interleaved, with each value computation of an
object being taken from the last side effect on that object in that
interleaving. This is normally referred to as "sequential consistency".
...
\end{note}
\end{quote}

Add a new paragraph after \CppXref{intro.multithread.general} paragraph 21:

\begin{quote}
\setcounter{Paras}{21}
\pnum
\begin{note}
The following holds for a data-race-free program: If the start
of an atomic block $T$ is sequenced before an evaluation $A$, $A$ is sequenced
before the end of $T$, $A$ strongly happens before some evaluation $B$, and $B$
is not sequenced before the end of $T$, then the end of $T$ strongly happens
before $B$. If an evaluation $C$ strongly happens before that evaluation $A$
and $C$ is not sequenced after the start of $T$, then $C$ strongly happens
before the start of $T$. These properties in turn imply that in any simple
interleaved (sequentially consistent) execution, the operations of each
atomic block appear to be contiguous in the interleaving.
\end{note}
\end{quote}

\rSec3[intro.progress]{Forward progress} 

Change in \CppXref{intro.progress} paragraph 1 as indicated:

\begin{quote}
\pnum
\removed{The implementation may assume that any thread will eventually do}
\added{An \defn{inter-thread side effect} is} one of the following:

\begin{itemize}
\tightlist
\item
  \removed{terminate,}
\item
  a call to a library I/O function,
\item
  an access through a volatile glvalue, or
\item
  a synchronization operation or an atomic operation
  \added{(\cxxref{atomics})}.
\end{itemize}

\added{The implementation may assume that any thread will eventually
terminate or evaluate an inter-thread side effect.}
\begin{note}
This is intended to allow compiler transformations
such as removal of empty loops, even when termination cannot be proven.
\end{note}
\end{quote}

\setcounter{chapter}{7}

\rSec0[stmt.stmt]{Statements}
\rSec1[stmt.pre]{Preamble}  
\pnum
Add a production to the grammar in \CppXref{stmt.pre} as indicated:

\begin{quote}
\begin{bnf}
\nontermdef{statement}\br
    labeled-statement\br
    \opt{attribute-specifier-seq} expression-statement\br
    \opt{attribute-specifier-seq} compound-statement\br
    \opt{attribute-specifier-seq} selection-statement\br
    \opt{attribute-specifier-seq} iteration-statement\br
    \opt{attribute-specifier-seq} jump-statement\br
    declaration-statement\br
    \opt{attribute-specifier-seq} try-block\br
    \added{atomic-statement}
\end{bnf}
\end{quote}

\setcounter{section}{7}

Add a new subclause before \CppXref{stmt.dcl}:

\begin{quote}
\rSec1[stmt.tx]{Atomic statement}  

\begin{bnf}
\nontermdef{atomic-statement}\br
    \keyword{atomic} \keyword{do} compound-statement
\end{bnf}

\pnum
An \grammarterm{atomic-statement} is also called an \emph{atomic block}.

\pnum
The start of the atomic block is immediately after the
opening \tcode{\{} of the \grammarterm{compound-statement}.
Evaluation of the end of the atomic block occurs
when the invocation of the destructor for a hypothetical automatic
variable of class type declared at the start of the atomic block would
happen (\cxxref{stmt.jump.general}).

\begin{note}
Thus, variables with automatic storage duration
declared in the \grammarterm{compound-statement}
are destroyed prior to reaching the end of the atomic block;
see \cxxref{stmt.jump}.
\end{note}

\pnum
A \keyword{case} or \keyword{default} label
appearing within an atomic block shall be associated with a
\keyword{switch} statement (\cxxref{stmt.switch}) within the same
atomic block. A label (\cxxref{stmt.label}) declared in an atomic block
shall only be referred to by a statement in the same atomic block.

\pnum
If the execution of an atomic block evaluates an inter-thread side
effect\iref{intro.progress} or if an atomic block is exited
via an exception, the behavior is undefined.

\pnum
\recommended
In case an atomic block is exited via an exception,
the program should be terminated
without invoking a terminate handler (\cxxref{exception.terminate}) or
destroying any objects
with static or thread storage duration (\cxxref{basic.start.term}).

\pnum
If the execution of an atomic block evaluates any of the following
outside of a manifestly constant-evaluated context (\cxxref{expr.const}),
the behavior is implementation-defined:

\begin{itemize}
\item
  an \grammarterm{asm-declaration} (\cxxref{dcl.asm});
\item
  an invocation of a function, unless
  \begin{itemize}
  \tightlist
  \item
    {the function is inline with a reachable definition or}
  \item
    {the function is a library function that may be used in an atomic
    block \ref{atomic.use};}
  \end{itemize}

\item
  a virtual function call (\cxxref{expr.call});
\item
  a function call, unless overload resolution selects

  \begin{itemize}
  \item
    a named function (\cxxref{over.call.func}) or
  \item
    a function call operator (\cxxref{over.call.object}),
    but not a surrogate call function;
  \end{itemize}
\item
  a \keyword{co_await} expression (\cxxref{expr.await}), a
  \grammarterm{yield-expression} (\cxxref{expr.yield}), or a
  \keyword{co_return} statement (\cxxref{stmt.return.coroutine});
\item
  dynamic initialization of a block-scope variable with static storage
  duration; or
\item
  dynamic initialization of a variable with thread storage duration.
  {{[} Note: That includes the case when such an initialization is
  evaluated within an atomic block because the initialization was
  deferred (\cxxref{basic.start.dynamic}). -\/- end note {]}}

\end{itemize}

\begin{note}
The implementation can define that the behavior is
undefined in some or all of the cases above.
\end{note}

\begin{example}
\begin{codeblock}
unsigned int f()
{
  static unsigned int i = 0;
  atomic do {
    ++i;
    return i;
  }
}
\end{codeblock}

Each invocation of \tcode{f}
(even when called from several threads simultaneously)
retrieves a unique value (ignoring wrap-around).
\end{example}

\begin{note}
Atomic blocks are likely to perform best where they execute
quickly and touch little data.
\end{note}
\end{quote}

\setcounter{chapter}{14}

\rSec0[cpp]{Preprocessor}

\setcounter{section}{10}

\rSec1[cpp.predefined]{Predefined macro names}

Add a row to Table \cxxref{tab:cpp.predefined.ft} in \CppXref{cpp.predefined}:

\setcounter{table}{18}

\begin{floattable}{Feature-test macros}{tab:intro.features}
{ll}
\topline
\lhdr{Macro name} & \rhdr{Value} \\
\capsep
\tcode{__cpp_transactional_memory}  & \tcode{\tsver} \\
\end{floattable}


\rSec0[library]{Library introduction}

\setcounter{section}{3}

\rSec1[requirements]{Library-wide requirements}

\setcounter{subsection}{6}
\setcounter{subsubsection}{16}

Add a new subclause after \CppXref{lib.types.movedfrom}:

\begin{quote}
\rSec3[atomic.use]{Functions usable in an atomic block}

\pnum
All library functions may be used in an atomic block\iref{stmt.tx}, except

\begin{itemize}
\item
  error category objects (\cxxref{syserr.errcat.objects})
\item
  time zone database (\cxxref{syserr.errcat.objects})
\item
  clocks (\cxxref{time.clock})
\item
  \tcode{signal} (\cxxref{support.signal}) and \tcode{raise}
  (\cxxref{csignal.syn})
\item
  \tcode{set_new_handler}, \tcode{set_terminate},
  \tcode{get_new_handler}, \tcode{get_terminate}
  (\cxxref{handler.functions}, \cxxref{alloc.errors}, \cxxref{exception.syn})
\item
  \tcode{system} (\cxxref{cstdlib.syn})
\item
  startup and termination (\cxxref{support.start.term}) except
  \tcode{abort}
\item
  \tcode{shared_ptr} (\cxxref{util.smartptr.shared}) and
  \tcode{weak_ptr} (\cxxref{util.smartptr.weak})
\item
  \tcode{synchronized_pool_resource} (\cxxref{mem.res.pool})
\item
  program-wide \tcode{memory_resource} objects
  (\cxxref{mem.res.global})
\item
  \tcode{setjmp} / \tcode{longjmp} (\cxxref{csetjmp.syn})
\item
  parallel algorithms (\cxxref{algorithms.parallel})
\item
  \tcode{random_device} (\cxxref{rand.device})
\item
  \tcode{locale} construction (\cxxref{locale.cons})
\item
  {\tcode{locale::global} (\cxxref {locale.statics})}
\item
  input/output (\cxxref{input.output})
\item
  atomic operations (\cxxref{atomics})
\item
  thread support (\cxxref{thread})
\end{itemize}
\end{quote}
